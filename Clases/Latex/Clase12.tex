\documentclass{article}
\usepackage{amsmath}
\usepackage{amssymb}
\usepackage{graphicx}
\usepackage{enumitem}
\usepackage{xcolor}
\usepackage[utf8]{inputenc}
\graphicspath{{}}	


%Aquí inicia la portada del documento
\title{\Huge\item\color{purple}\textit{BITÁCORA DE LA CLASE NÚMERO 12 DEL CURSO}}
\author{\Large\textit{ Mariana Yasmin Martínez García}}
\date{\Large\textit{ 21/01/2019}}

\begin{document}
\begin{figure}[t]
	\centering
	\includegraphics[width=0.8\linewidth]{Imagenes/1}
	\caption{Escudo de la Facultad de ciencias}
	\label{fig:1}
\end{figure}
	\maketitle
		
	\newpage
	
	\title{\Huge\textbf{\color{purple}\textit{Bitácora: 22 de enero\\}}} \\
	En esta clase lo primero que hicimos fue continuar con lo que no habíamos terminado una clase anterior que era hacer por comprensión una tabla que icluyera ordenadamente los grados Celsius y su correspondiente grado Fahreheit. \\
	Cuando terminamos eso, hicimos un ejercicio que tenía listas dentro de una lista y debíamos de mostrar el contendido de las listas interiores de ua forma ordenada e inicialmente usando índices:\\
	\begin{verbatim}
	for i in range(len(alumnos)):
		for j in range(len(alumnos[i])):
			calificacion = alumnos[i][j]
			print '%4d' % calificacion,
		print 
	\end{verbatim}
	Aquí usamos algo que no habíamos usado y esto fue el uso de la coma (,) al final de "calificación" para poder separar cada una de las listas por un renglón. \\
	Después hicimos lo mismo solo que esta vez no usamos \textbf{range}, y quedó así: \\
	\begin{verbatim}
	for lista in alumnos:
		for calificacion in lista:
			print '%4d' % calificacion,
		print
	\end{verbatim} 
Donde "alumnos" es el nombre de la lista, "lista" es el nombre que le asigné a cada una de las listas interiores y "calificación" son el contenido de las listas "lista". \\ 
	Vimos que al hacer una referencia de alguna lista al modificar cualquiera de ellas se modifica la otra. Sin embargo, si hacemos una copia de ella y la modificamos, la original no se modifica; o si modificamos la original, la copia no se modificará. \\ \\
	Al final de la clase usamos LaTex, esta vez usamos \textbf{documentclass{beamer}} que sirve para poder hacer una presentación tipo PowerPoint. \\
	Aprendimos a crear nuevas páginas, hacer transiciones y a cambiar el estilo del fondo.
\end{document}