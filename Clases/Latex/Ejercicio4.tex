\documentclass{beamer}
\usepackage{graphicx}
\usepackage[utf8]{inputenc}
%\usepackage[spanish]{babel}
\graphicspath{{Imagenes/}}
%\usetheme{Antibes}
%\usetheme{AnnArbor}
%\usetheme{Berkeley}
%\usetheme{CambridgeUS}
%\usetheme{Goettingen}
%\usetheme{Bergen}
\usetheme{Hannover}

%Esto únicamente se utiliza para el tema Bergen
%\def\insertauthorindicator(¿Quién?)
%\def\insertdateindicator(Fecha)



\title{Taller de Herramientas Computacionales}
\author{Mariana Yasmin Martínez García}
\date{\today}

\begin{document}
%	\maketitle
	
\begin{frame}
%\transblinshorizontal
	\frametitle{Mi primera presentación en LaTex}
	\includegraphics[scale=0.50]{1}
\end{frame}

\begin{frame}
	\frametitle{Segunda diapositiva}
	Esta es mi segunda diapositiva
\end{frame}

\begin{frame}[fragile] {Tercera diapositiva} %fragile te permite introducir caractéres especiáles
	\begin{verbatim}
	x = 10.5; y = 1.0/3; z = 15.3
	# Esta es otra forma de escribir las variables: x,y,z=10.5,1.0/3,15.3
	H = '''
	El punto en R3 es:
	(x,y,z)=(%.2f,%g,%G)
	''' % (x,y,z)
	print H
	
	G = '''
	El punto en R3 es:
	(x,y,z)=({laX:.2f},{laY:g},{laZ:G}
	'''.format(laX=x,laY=y,laZ=z)
	print G
	\end{verbatim}
\end{frame}

\end{document}