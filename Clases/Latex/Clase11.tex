\documentclass{article}
\usepackage{amsmath}
\usepackage{amssymb}
\usepackage{graphicx}
\usepackage{enumitem}
\usepackage{xcolor}
\usepackage[utf8]{inputenc}
\graphicspath{{}}	


%Aquí inicia la portada del documento
\title{\Huge\item\color{purple}\textit{VITÁCORA DE LA CLASE NÚMERO 11 DEL CURSO}}
\author{\Large\textit{ Mariana Yasmin Martínez García}}
\date{\Large\textit{ 21/01/2019}}

\begin{document}
\begin{figure}[t]
	\centering
	\includegraphics[width=0.8\linewidth]{Imagenes/1}
	\caption{Escudo de la Facultad de ciencias}
	\label{fig:1}
\end{figure}
	\maketitle
		
	\newpage
	
	\title{\Huge\textbf{\color{purple}\textit{Vitácora: 21 de enero\\}}} \\
	En esta clase revisamos dudas sobre la tarea 5 donde debíamos involucrar listas para poder resolverlos. \\
	Empezamos haciendo de diferentes maneras que el usuario decidiera cuántos valores quería usar y que el programa repitiera cierta acción hasta que tuviera esos valores:
	\begin{verbatim}
	n= input(¿Cuántos valores?)
	
	Esta es una forma de hacerlo:
	for i in range(n):
	valor = input("Escribe el valor: ")
	L.append(valor)
	
	Esta es otra:
	M= range(n)
	for i in M:
	valor = input("Dame el valor: ")
	M[1](valor)
	
	Esta es otra forma:
	N = range(n)
	j = 0
	while j < n:
	valor = input("Dame el número: ")
	N[1](valor)
	\end{verbatim}
Esto me sirvió muchísimo porque yo había intentado algo parecido pero no había podido. \\ \\
Después de hacer eso hicimos un programa que nos permitía imprimir Grados Celsius desde cierto límite izquierdo hasta algún límite del lado derecho con su respectivo grado Fahrenheit en forma de tabla para que facilitara la lectura y comprensión. \\
Y terminamos haciendo algo parecido pero ahora usando algo llamado \textbf{zip} que te permite combinar ambas listas (la de Celsius y Fahrenheit).

\end{document}