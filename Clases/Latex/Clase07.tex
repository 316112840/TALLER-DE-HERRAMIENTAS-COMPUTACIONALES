\documentclass{article}
\usepackage{amsmath}
\usepackage{amssymb}
\usepackage{graphicx}
\usepackage{enumitem}
\usepackage{xcolor}
\usepackage[utf8]{inputenc}
\graphicspath{{}}	


%Aquí inicia la portada del documento
\title{\Huge\item\color{blue}\textbf{BITÁCORA DEL DÍA NÚMERO 07 DEL CURSO}}
\author{\Large Mariana Yasmin Martínez García}
\date{\Large 16/01/2019}

\begin{document}
\begin{figure}[t]
	\centering
	\includegraphics[width=0.4\linewidth]{Imagenes/1}
	\caption{Escudo de la Facultad de ciencias}
	\label{fig:1}
\end{figure}
	\maketitle
		
	\newpage
	
	\title{\huge\textbf{Bitácora del día 15 de enero\\}} \\
	Este día empezamos retomando el programa de la clase pasada que no habíamos terminado(el que sirve para calular la raíz cuadrada). La clase pasada definimos una función para poder calcular el valor absoluto y en esta nueva función a la que llamamos \textbf{raiz} usamos esa función pasada. \\
	Lo primero que hicimos fue asignar valores iniciales, cuando hicimos esto el profesor nos explico que un signo de igual ("=") sirve para asignar y dos ("==") sirven para denotar una igualdad. \\
	Y,como mencioné en la vitácora pasada, el proceso o las funciones debían repetirse hasta que el rango de error (en este caso denotado por la letra \textbf{e} y que le asignamos un valor de 0.0001) se cumpliera. Para hacer esto usamos la palabra \textbf{while} seguido de \textbf{vAbsoluto(b-h)>e} donde \textbf{vAbsoluto} es el nombre de la función para calcular el valor absoluto que hicimos la clase pasada. Mientras el valor absoluto de b-h se cumpliera,entonces las dos fórmulas que ya habíamos pensado un día anterior se iban a seguir repitiendo hasta que la condición de ya no ocurriera. Y por último colocamos la palabra \textbf{return} para que de esta manera nos regresara el resultado que es la raíz cuadrada. \\ \\
	Cuando terminamos ese, el profesor nos pidió que hiciéramos el mismo solo que ahora también nos dijera la cantidad de veces que se repitió el proceso para llegar al resultado (la raíz), y para lograr esto a otra variable (\textbf{i}) le asignamos el valor \textbf{0}(ya que cuando se empezara el programa,la cantidad de veces que se había repetido el programas eran 0) y después, en el bloque de \textbf{while} colocamos esta fórmula(\textbf{i = i + 1}, que lograba que cada vez que se repetía el ciclo fuera aumentando de 1 en 1. Y para mostrar el resultado escribimos "\textbf{print"El ciclo se repitió \% veces" \%(i)}" Esto permitía que donde está escrito el primer \% se colocara el número de veces al que \textbf{i} había llegado para obtener la condición, es decir, el número de veces que se repitió el ciclo. \\ \\
	Cuando terminamos esos programas el profesor nos dejó hacer otro que consistía en que si el número (x) que se le daba al programa era par debía pasar x/2 y si el número era impar, \textbf{3*x + 1} y se debía repetir el ciclo hasta que el resultado fuera \textbf{1} y no solo eso, también debía de decir la cantidad de veces que se repitió el ciclo para que \textbf{x} fuera igual a 1. \\
	Mi archivo con mi versión de este programa se llama \textbf{Ulam0} y está guardado en la carpeta de programas.\\ \\
	Lo último que vimos esta clase fue a introducir y escribir símbolos matemáticos en documentos de Látex, lo cuál me ayudó para corregir algunas de la vitácoras pasadas y para que este documento se viera mejor.\\
	Aprendimos a colocar el símbolo de la raíz cuadrada, colocar fracciones, integrales y a como colocar el \% sin que me lo marque como comentario.


\end{document}