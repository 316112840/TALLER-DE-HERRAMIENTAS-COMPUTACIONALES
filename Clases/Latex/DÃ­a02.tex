\documentclass[letterpaper, 12pt, twoside]{article}
\usepackage{amsmath}
\usepackage{graphicx}
\usepackage{xcolor}
\usepackage{enumitem}
\usepackage[utf8]{inputenc}
\usepackage{graphicx}

%Aquí inicia la portada del documento
\title{\Huge\item\color{red}\textbf {VITÁCORA DEL DÍA NÚMERO 02 DEL CURSO}}
\author{Mariana Yasmin Martínez García}
\date{13/01/2019}

\begin{document}
	\maketitle
	
	\newpage
	\title{\huge\textbf{Vitácora del día 8 de enero\\}}
	El día 2 del curso nos cambiamos de salón al salón de computo 1 por lo que a partir de aquí empezamos a usar la computadora, este día se revisaron más comandos:
	\begin{enumerate}
		\item \textbf{\large set }: se usa para ver cuáles son las variables de entorno.
		\item \textbf{\large less }: muestra el contenido de forma paginada.
		
	\end{enumerate}
    Y empezamos a usar un servicio llamado Github el cual nos servirá para poder guardar en la nube nuestros programas y archivos de Látex.
    Por lo tanto creamos nuestra cuenta en esta plataforma y en la página creamos un repositorio. \\
    Para poder instalar \textbf{git} usamos \textbf{"sudo apt-get install"} para máquinas con Ubuntu y \textbf{"sudo yum install git"} para las que contienen Fedora. Se nos enseñó a usar comandos para realizar acciones en \textbf{git}, por ejemplo:
    \begin{enumerate}
    	\item\textbf{\large git clone}:se escribe este comando seguido del link del repositorio que creamos, de esta manera se hace una carpeta en nuestra computadora que será nuestro repositorio. 
    	\item\textbf{\large gitconfig --global user.email ""} y \textbf{ \large gitconfig --global user.name ""}: que sirven para introducir tu nombre de usuario y el email para de esa manera poder trabajar en \textbf{git} desde la terminal.
    	\item\textbf{\large git add * }: sirve para registrar los cambios de los archivos y/o la creación de nuevos.
    	\item\textbf{\large git commit}: con este comando se te abrirá una pestaña en \textbf{vi} que te permitirá colocar un comentario cuando subas los archivos a la nube.
    	\item\textbf{\large git push}: sirve para subir los archivos nuevos o las modificaciones a la nube. Al colocar este comando te pedirá que escribas tu nombre de usurio y contraseña.
    \end{enumerate}

\end{document}