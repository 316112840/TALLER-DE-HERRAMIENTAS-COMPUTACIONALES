\documentclass{article}
\usepackage{amsmath}
\usepackage{amssymb}
\usepackage{graphicx}
\usepackage{enumitem}
\usepackage[utf8]{inputenc}
\graphicspath{{Imagenes}}	
\title{\Huge Taller de Herramientas Computacionales}
\author{Mariana Yasmin Martínez García}
\date{15 de enero del 2019}
\begin{document}
	\maketitle
	\begin{figure}
		\includegraphics[scale=.5]{Imagenes/1}
		\caption{Escudo de la Facultad de ciencias}
		\label{fig:1}
	\end{figure}	
	\newpage
	%\section{Expresiones Matemáticas}
	\section*{Expresiones Matemáticas}
	$ \alpha + \beta $ \\  % En lugar de los símbolos de peso podemos usar:  \(\)
	\[ \alpha + \beta \] %Este lo centra	
	\section*{Índices y subíndices}
	$X_{2}$ \\
	$X^{2}$  \\	
	\section*{Fracciones}
	$\frac{3}{5}$\\ \\
	$\frac{\frac{3}{4}}{\frac{2}{3}}$ \\ \\	
	\section*{Raíz}
	$\sqrt{2} + \sqrt{3^2}^5$ \\ \\ 
	\section*{Integrales}
	$\int_{a}^{b} x^2 dx$ \\ \\
	$\int_{a}^{b} x^2 \partial x^2$ \\ \\
	\section*{Para colocar espacios}
	$3 \quad 2$
	\section*{Matrices}
	\[
	\begin{Bmatrix}
		x_{2} & x_{3}\\
		x_{4} & x_{6}
	\end{Bmatrix}
	\]
	
	\[  
	\begin{Bmatrix}
	x_{2} & x_{5} & \dots\\
	x_{5} & x_{20} & \ddots\\
	\vdots & \vdots & \vdots
	\end{Bmatrix}
	\]
	
	$\sum$ \\ \\ \\
	
	
	$\ddots$\\
	\vdots\\
	\dots \\ \\
	\% %para que no se tome como comentario
\end{document}
