\documentclass{article}
\usepackage{amsmath}
\usepackage{amssymb}
\usepackage{graphicx}
\usepackage{enumitem}
\usepackage{xcolor}
\usepackage[utf8]{inputenc}
\graphicspath{{Imagenes}}	


%Aquí inicia la portada del documento
\title{\Huge\item\color{orange}\textbf{PROBLEMA 3 }}
\author{\Large Mariana Yasmin Martínez García}
\date{\Large 20/01/2019}


\begin{document}

\begin{figure}[t]
	\centering
	%\includegraphics[width=0.7\linewidth]{home/mariana/Documentos/TALLER-DE-HERRAMIENTAS-COMPUTACIONALES/Clases/Latex/Imagenes/1}
	\caption{}
	\label{fig:1}
\end{figure}





	\maketitle
		
	\newpage
	
	\title{\huge\textbf{Problema 3: Convertir de grados C a  grados F y viceversa }} \\
	En este primero estaba pensando en cómo escribir una función que me permitiera hacer ambas pero finalmente decidí hacer dos; una que me permitiera transformar de grados Celsius a Fahrenheit y otra que fuera de grados Fahrenheit a Celsius. \\
	Lo único que hice fue buscar la fórmula que relaciona ambas fórmulas y esta es: \[ F = \frac{9}{5} C +32 \]
	Con esta fórmula lo único que hice fue:\\
	def CeFa(c): \\
		f = c*9/5 +32\\
		return f\\
	Definí una función que va de grados C a grados F,y que te regresa "f" que son los grados Fahrenheit. Y para transformar de grados Fahrenheit a Celsius solo despejé a la "c".\\
	Con este programa tampoco tuve problemas, aunque siento que pude haber hecho mucho mejor la parte correspondiente a "Problema3S.py"


\end{document}