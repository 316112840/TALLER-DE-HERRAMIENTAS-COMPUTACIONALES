\documentclass{article}
\usepackage{amsmath}
\usepackage{amssymb}
\usepackage{graphicx}
\usepackage{enumitem}
\usepackage{xcolor}
\usepackage[utf8]{inputenc}
\graphicspath{{/home/mariana/Documentos/TALLER-DE-HERRAMIENTAS-COMPUTACIONALES/Clases/Latex/Imagenes}}	


%Aquí inicia la portada del documento
\title{\Huge\item\color{orange}\textbf{PROBLEMA 7 }}
\author{\Large Mariana Yasmin Martínez García}
\date{\Large 20/01/2019}


\begin{document}

\begin{figure}[t]
	\centering
	\includegraphics[width=0.7\linewidth]{{/home/mariana/Documentos/TALLER-DE-HERRAMIENTAS-COMPUTACIONALES/Clases/Latex/Imagenes/1}}
	\caption{Escudo de la Facultad de ciencias}
	\label{fig:1}
\end{figure}


	\maketitle
		
	\newpage
	
	\title{\huge\textbf{Problema 7: Dado 10 datos indicar el mayor, el menor y su promedio }} \\
	Con este tuve un poco de problema ya que yo tenía en mente comparar los todos juntos. Así que estuve pensando por mucho tiempo cómo lograr una función que me permitiera hacer eso y después de mucho tiempo me dí cuenta de que de esa manera no iba a lograr nada, ya que para ver si uno es mayor a los demás, ese debe ser mayor a cada uno de los otros, por lo que debía de comparar de dos en dos. \\
	Para comparar el número "a" con el número "b" hice esto:
	\[if a>b:\]
	\[tmp = a\]
	\[a = b\]
	\[b = tmp\]
	Esto lo hice retomando algo que vimos en clase. Hcaiendo esto estaba intercambiando los valores "a" y "b" hasta que a fuera el menor de los dos, y así seguí haciendo lo con los demás números. Cuando "a" era el menor de entre "a" y "b" ahora comparaba ese "a" con "c" y así sucesivamente hasta haberlos comparado todos. Así "a" sería el más pequeño de cada uno de ellos. \\
	Hice casi lo mismo para saber cuál era el mayor, solo que ahora en lugar de hacer que "a" fuera el mayor, debía de hacer que "a" fuera el mayor:
	\[if a<b:\]
	\[tmp = a\]
	\[a = b\]
	\[b = tmp\]
	Y ya, para el promedio usé la fórmula del problema 6.



\end{document}