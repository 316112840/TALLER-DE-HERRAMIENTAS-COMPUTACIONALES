\documentclass{article}
\usepackage{amsmath}
\usepackage{amssymb}
\usepackage{graphicx}
\usepackage{enumitem}
\usepackage{xcolor}
\usepackage[utf8]{inputenc}
\graphicspath{{Imagenes}}	


%Aquí inicia la portada del documento
\title{\Huge\item\color{orange}\textbf{PROBLEMA 2 }}
\author{\Large Mariana Yasmin Martínez García}
\date{\Large 20/01/2019}


\begin{document}

\begin{figure}[t]
	\centering
	\includegraphics[width=0.7\linewidth]{../Imagenes/1}
	\caption{}
	\label{figura:1}
\end{figure}



	\maketitle
		
	\newpage
	
	\title{\huge\textbf{Problema 2: Calcular los tiempos donde se alcanza cierta altura dada }} \\
	Bueno, este es solo una modificación de un problema que ya habíamos trabajado en clase solo que en ese debíamos calcular la altura dados el tiempo, y en este debemos calcular los tiempos dados la altura. Así que lo que se me ocurrió fue cambiar la fórmula, de: 	\[ y ={v_{0}} t - \frac{1}{2}g t^{2}  \] despejé el tiempo y las fórmulas quedaron así:
	\[ t1 = \frac{v_{0} + \sqrt(v_{0}^2 - 2ah)}{a} \]
	\[ t2 = \frac{v_{0} - \sqrt(v_{0}^2 - 2ah)}{a} \]
	A la gravedad le asigné el valor 9.81 y a la $v_0$ y a la altura son las que dependerán del usuario por lo que no leas asigné ni un valor. Al final solo hice que imprimiera ambos tiempos y ya. \\
	Con este problema no tuve dificultades.


\end{document}