\documentclass{article}
\usepackage{amsmath}
\usepackage{amssymb}
\usepackage{graphicx}
\usepackage{enumitem}
\usepackage{xcolor}
\usepackage[utf8]{inputenc}
\graphicspath{{TALLER-DE-HERRAMIENTAS-COMPUTACIONALES/Clases/Latex/Imagenes}}	


%Aquí inicia la portada del documento
\title{\Huge\item\color{orange}\textbf{PROBLEMA 6 }}
\author{\Large Mariana Yasmin Martínez García}
\date{\Large 20/01/2019}


\begin{document}

\begin{figure}[t]
	\centering
	\includegraphics[width=0.7\linewidth]{../Imagenes/1}
	\caption{}
	\label{figura:1}
\end{figure}


	\maketitle
		
	\newpage
	
	\title{\huge\textbf{Problema 6: Calcular el promedio de 10 números dados }} \\
	Con este se me ocurrió definir 10 variables diferentes, cada una de ellas sería cada unos de los 10 números dados. Después definí otra variable que sería igual a la suma de todas las otras variables y las dividí entre 10:
	\[x = \frac{a+b+c+d+e+f+g+h+i+j}{10}\]
	Solo usé la fórmula para calcular promedios y al final sólo hice que me imprimiera el valor final de "x".\\
	Con este tampoco tuve problemas.


\end{document}