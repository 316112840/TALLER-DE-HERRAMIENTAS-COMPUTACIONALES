\documentclass[letterpaper, 12pt, twoside]{article}
\usepackage{amsmath}
\usepackage{graphicx}
\usepackage{xcolor}
\usepackage{enumitem}
\usepackage[utf8]{inputenc}
\usepackage{graphicx}

%Aquí inicia la portada del documento
\title{\Huge\item\color{red}\textbf {VITÁCORA DEL DÍA NÚMERO 05 DEL CURSO}}
\author{Mariana Yasmin Martínez García}
\date{13/01/2019}

\begin{document}
	\maketitle
	
	\newpage
	\title{\huge\textbf{Vitácora del día 11 de enero\\}} \\
	El día 11 de enero del 2019 retomamos el ejemplo del profesor del día anterior para poder crear un \textbf{módulo} (también conocido como \textbf{biblioteca}) para que de esta manera podamos llamarlo después y nos permita resolver un problema similar pero con otros valores.\\
	Mientras definíamos la fórmula para resolver el problema el profesor  mencionó \textbf{"palabras reservadas"}, nos explicó que son palabras que tienen un significado dentro de \textbf{python} y no podemos utilizar las como variables, ejemplos de estas son: \textbf{print} o \textbf{import}. \\
	A partir de ahorita donde escriba \textbf{-} querrá decir que debe haber un signo de porcentaje:\\ \\
	El profesor escribió: \textbf{print 'La posición de la pelota en el t=-g es -.2f' - (t,y)}
	y nos explico que donde está el primer \textbf{-} se refiere al \textbf{t} y el segundo es \textbf{y}. Donde la g y f quieren decir que son de tipo flotante.
	También vimos muchas otras formas de poner el \textbf{print}, las cuales están guardadas en un archivo en la carpeta Programas llamado EjemplosPrint.py \\  \\
	Al terminar de difinir nuestra función para solucionar el problema habíamos hecho un módulo por lo que debíamos llamarlo cuando quisiéramos resolver lo y esto se hace con el comando \textbf{import} seguido del nombre del programa donde habíamos hecho la fórmula. \\
	Después de importarlo bastaba con escribir el nombre del archivo, escribir un punto y presionar el tabulador para que pudieras escoger la opción que te permitiera no solo escoger los valores que te había proporcionado ese problema en particular sino cualquiera otros valores.\\ \\
	Lo último que hicimos esta clase fue usar \textbf{Latex}, nos enseñaron la función de algunos paquetes como \textbf{usepackage[utf8]{inputenc}} que nos permite usar acentos y diéresis. También vimos cómo crear una portada sencilla y cómo cambiar el color de las palabras y cómo hacer más gruesa algunas palabras. \\
	Y también nos enseñaron a hacer una lista  y enumerar.
	Se nos dejó de tarea hacer estos resúmenes usando lo visto en clase.
	
	

\end{document}