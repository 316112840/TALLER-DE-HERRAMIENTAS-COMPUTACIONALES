\documentclass[letterpaper, 12pt, twoside]{article}
\usepackage{amsmath}
\usepackage{graphicx}
\usepackage{xcolor}
\usepackage{enumitem}
\usepackage[utf8]{inputenc}
\usepackage{graphicx}

%Aquí inicia la portada del documento
\title{\Huge\item\color{red}\textbf {VITÁCORA DEL DÍA NÚMERO 05 DEL CURSO}}
\author{Mariana Yasmin Martínez García}
\date{13/01/2019}

\begin{document}
	\maketitle
	
	\newpage
	\title{\huge\textbf{Vitácora del día 11 de enero\\}} \\
	El día 11 de enero del 2019 retomamos el ejemplo del profesor del día anterior para poder crear un \textbf{módulo} (también conocido como \textbf{biblioteca}) para que de esta manera podamos llamarlo después y nos permita resolver un problema similar pero con otros valores.\\
	Mientras definíamos la fórmula para resolver el problema el profesor  mencionó \textbf{"palabras reservadas"}, nos explicó que son palabras que tienen un significado dentro de \textbf{python} y no podemos utilizar las como variables, ejemplos de estas son: \textbf{print} o \textbf{import}. \\
	A partir de ahorita donde escriba \textbf{-} querrá decir que debe haber un signo de porcentaje:\\ \\
	El profesor escribió: \textbf{print 'La posición de la pelota en el t=-g es -.2f' - (t,y)}
	y nos explico que donde está el primer \textbf{-} se refiere al \textbf{t} y el segundo es \textbf{y}. Donde la g y f quieren decir que son de tipo flotante.
	También vimos muchas otras formas de poner el \textbf{print}. 
	
	

\end{document}