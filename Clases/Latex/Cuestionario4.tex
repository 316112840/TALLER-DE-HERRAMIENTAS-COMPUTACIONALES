\documentclass{article}
\usepackage{amsmath}
\usepackage{amssymb}
\usepackage{graphicx}
\usepackage{enumitem}
%\usepackage{xcolor}
\usepackage[dvipsnames]{xcolor}
\usepackage[utf8]{inputenc}
\graphicspath{{}}	


%Aquí inicia la portada del documento
\title{\Huge\item\color{magenta}\textit{PREGUNTAS DE LA CLASE 13 DEL CURSO}}
\author{\Large\textit{Mariana Yasmin Martínez García}}
\date{\Large\textit{23/01/2019}}


\begin{document}
\begin{figure}[t]
	\centering
	\includegraphics[width=0.7\linewidth]{Imagenes/1}
	\caption{Escudo de la Facultad de ciencias}
	\label{figura:1}
\end{figure}

	\maketitle
		
	\newpage
	
	\title{\huge\color{magenta}\textbf{\textit{Cuestionario de la clase 13}}} \\
	\begin{enumerate}
		 \item{\Large\color{purple} ¿Qué es una función recursiva?} una función recursiva es en la que dentro de ella se menciona a sí misma.
		 \item{\Large\color{purple} ¿Qué necesita una función para que sea recursiva?} necesita al menos un "if" y llamarse a sí misma.
		 \item{\Large\color{purple} Menciona lo dos tipos de variables que hay: } variables locales  variables globales.
		 \item{\Large\color{purple} ¿De qué manera un objeto se toma como verdadera?} cuando el objeto o es vacío o es diferente de 0.
		 \item{\Large\color{purple} ¿Cómo se llama la función que contiene \textbf{return}?} procedimientos
		 \item{\Large\color{purple} ¿Qué es \textbf{global} y para qué sirve?} es una palabra reservada y sirve para definir a alguna variable como global  de esa manera si la variable estaba dentro de un bloque poder usarla fuera de él. s
	\end{enumerate}
	
\end{document}