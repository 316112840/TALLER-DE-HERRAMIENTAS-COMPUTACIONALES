\documentclass{article}
\usepackage{amsmath}
\usepackage{amssymb}
\usepackage{graphicx}
\usepackage{enumitem}
%\usepackage{xcolor}
\usepackage[dvipsnames]{xcolor}
\usepackage[utf8]{inputenc}
\graphicspath{{}}	


%Aquí inicia la portada del documento
\title{\Huge\item\color{magenta}\textit{PREGUNTAS DE LA CLASE 12 DEL CURSO}}
\author{\Large\textit{Mariana Yasmin Martínez García}}
\date{\Large\textit{21/01/2019}}


\begin{document}
\begin{figure}[t]
	\centering
	\includegraphics[width=0.7\linewidth]{Imagenes/1}
	\caption{Escudo de la Facultad de ciencias}
	\label{figura:1}
\end{figure}

	\maketitle
		
	\newpage
	
	\title{\huge\color{magenta}\textbf{\textit{Cuestionario de la clase 11}}} \\
	\begin{enumerate}
		 \item{\Large\color{purple} ¿Para qué sirve "enumerate"?} te muestra el índice y el valor de ese índice en la lista.
		 \item{\Large\color{purple} ¿Qué es un "tuple"?} una tupla es una lista que no se puede cambiar y a diferencia de las listas, las tuplas usan paréntesis y no corchetes.
		 \item{\Large\color{purple} ¿Para qué sirve "zip"?} zip te permite unir varias listas y reorganizarlas en una tupla. Tomará el iésimo término  de cada lista y las irá colocando en esta nueva lista.
		 \item{\Large\color{purple} ¿Para qué sirve "float"?} para permitir que algún resultado sea flotante y de esa manera que te pueda mostrar decimales.
	\end{enumerate}
	
\end{document}