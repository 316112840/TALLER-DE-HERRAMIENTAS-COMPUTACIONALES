\documentclass{article}
\usepackage{amsmath}
\usepackage{amssymb}
\usepackage{graphicx}
\usepackage{enumitem}
\usepackage{xcolor}
\usepackage[utf8]{inputenc}
\graphicspath{{}}	


%Aquí inicia la portada del documento
\title{\Huge\item\color{blue}\textbf{BITÁCORA DEL DÍA NÚMERO 08 DEL CURSO}}
\author{\Large Mariana Yasmin Martínez García}
\date{\Large 16/01/2019}

\begin{document}
\begin{figure}[t]
	\centering
	\includegraphics[width=0.7\linewidth]{Imagenes/1}
	\caption{}
	\label{fig:1}
\end{figure}

	\maketitle
		
	\newpage
	
	\title{\huge\textbf{Bitácora del día 16 de enero\\}} \\
	Este día lo empezamos haciendo un repaso de todo lo que habíamos visto los días pasados. \\
	Usamos el comando top para poder saber información sobre el sistema operativo, núcleos,etc..\\
	Usamos el comando \textbf{chmod tx 08.py} para poder modificar los permisos de este archivo que creamos ese día. Con esto lo que hicimos fue hacer el archivo ejecutable. \\
	Esto lo pudimos corroborar con el comando \textbf{ls}. \\
	Antes de modificarlo el archivo aparecía en gris y al volverlo ejecutable y escribir el comando \textbf{ls} el archivo aparecía verde. \\
	Usamos el comando \textbf{find . -name "*.py"} para buscar todos los archivo que se encontraban en el directorio actual que termiaran en .py.\\ 
	También escribimos \textbf{whereis } seguido del nombre de algún archivo y esto lo que te permitía es saber donde se encuentra dicho archivo.\\ \\
	Después de eso através del programa python le dimos órdenes al Bash. Y esto fue gracias a que escribimos "\textbf{\#!/urs/bin/python2.7}" (con esto también denotamos que estábamos usando la versión 2.7 de Python.) \\
	Con este comando desde el Bash podíamos ejecutar el archivo de Python sin la necesidad de usar el Shell.\\ \\ 
	Después el profesor nos habló sobre clases, objetos y atributos. Los objetos pertenecen a las clases y los objetos tienen atributos.\\
	Después mencionó la palabra \textbf{método} y nos dijo que cada método está asociado a un objeto y tienen la forma: \textbf{objeto.método( )}. \\
	Este formato se parece a cuando usamos alguna biblioteca (o módulo) pero las bibliotecas NO son objetos.\\ \\
	Sobre el Shell que se abre al escribir \textbf{idle} sobre la Terminal usamos \textbf{type( )}, y te dirá de qué tipo es lo que sea que esté dentro del paréntesis. \\
	También podíamos escribir \textbf{str(5)*4} la respuesta del Shell sería \textbf{'5,5,5,5'}. Con \textbf{str} se refiere a que es de tipo cadena, por eso te regresa una cadena. \textbf{int} es de tipo entero. \textbf{float} es de tipo flotante.\\ \\ 
	Al final de la clase revisamos un poco de LaTex. Como varios tenían problemas con insertar imágenes en esta clase se volvió a revisar eso. \\
	Y por último se nos dejó de tarea hacer 10 programas usando lo que habíamos visto y revisado en todos estos días.
	


\end{document}