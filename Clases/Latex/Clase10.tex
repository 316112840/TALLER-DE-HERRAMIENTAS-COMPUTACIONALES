\documentclass{article}
\usepackage{amsmath}
\usepackage{amssymb}
\usepackage{graphicx}
\usepackage{enumitem}
\usepackage{xcolor}
\usepackage[utf8]{inputenc}
\graphicspath{{}}	


%Aquí inicia la portada del documento
\title{\Huge\item\color{blue}\textbf{BITÁCORA DEL DÍA NÚMERO 10 DEL CURSO}}
\author{\Large Mariana Yasmin Martínez García}
\date{\Large 18/01/2019}


\begin{document}
\begin{figure}[t]
	\centering
	\includegraphics[width=0.7\linewidth]{Imagenes/1}
	\caption{}
	\label{fig:1}
\end{figure}

	\maketitle
		
	\newpage
	
	\title{\huge\textbf{Bitácora del día 18 de enero}} \\
	La clase anterior nos enseñaron que \textbf{!=} es para denotar que algo no es igual, en esta clase retomamos eso y además vimos que \textbf{+=} es para sumar, por ejemplo: \textbf{a += b} es lo mismo que decir \textbf{a = a + b}. \\
	Escribir \textbf{a -= b} es lo mismo que \textbf{a = a - b} \\
	Escribir \textbf{a *= b} es igual a escribir \textbf{a = a*b} \\
	Y \textbf{a /= b } es lo mismo que \textbf{a = a/b} \\
	Si tuviéramos \textbf{a += b} esto quiere decir que el nuevo valor de "a" será el anterior valor de "a" más el valor de "b".\\
	También aprendimos que podemos usar \textbf{not} seguido de alguna igualdad o desigualdad para denotar que es lo contrario a eso,que estamos negando esa expresión. \\
	A una variable le asignamos como valor una cadena. \\
	Con \textbf{bool ( )} y colocando entre los paréntesis cualquier variable que ya hubiéramos asignado, nos diría si la variable contiene algún objeto respondiéndonos con "True". Si en cambio estuviera vacía, nos respondería "False" para decirnos que el falso. \\ \\
	En esta clase empezamos a usar listas. Las listas siempre están entre corchetes, por ejemplo: \[ L = [12, 10, 9]  \]
	Donde a la variable "L" le asignamos una lista. Una lista es vacía si la colocamos de la siguiente manera:  \[ L = []  \]
	Para introducir algún elemento a la lista, usamos \textbf{.append( )}, dentro de los paréntesis se coloca el objeto que queramos introducir y antes del punto se coloca el nombre de la lista a la que le queramos introducir el objeto. Este objeto que introduciremos se colocará al final de la lista. \\ \\
	Con \textbf{.insert( , )} se introduce también cualquier objeto, la diferencia con el anterior es que en este se puede especificar en qué lugar colocarlo. Antes de la coma (,) se pondrá el lugar en que queramos introducirlo y después de ella se colocará el objeto que se incluirá ala lista, el lugar se definirá con el índice, el objeto que se encuentra en la posición 1 está en el índice 0, el de la posición 2 se encuentra en el índice 1 y así sucesivamente.. Y como en el caso anterior, antes del punto se escribirá el nombre de la lista. \\ \\
	Con \textbf{.pop( )}, escribiendo el nombre de la lista antes del punto y entre paréntesis el índice de algún objeto, se sacará dicho objeto y lo mostrará. Si no se colocará ningún número dentro del paréntesis, sería los mismo que poner el número correspondiente al índice del último objeto de la lista. \\ \\
	Y por último usamos \textbf{.extend( )}, al igual que en los anteriores se colocará el nombre de la lista antes del punto, y entre los paréntesis se colocarán los objetos que se quieran y se incluirán a la lista. Solo que en este caso si se incluye una lista, cada uno de los objetos de esa lista se incluirán como objetos independientes a la otra lista.
	


\end{document}