\documentclass{article}
\usepackage{amsmath}
\usepackage{amssymb}
\usepackage{graphicx}
\usepackage{enumitem}
\usepackage{xcolor}
\usepackage[utf8]{inputenc}
\graphicspath{{}}	


%Aquí inicia la portada del documento
\title{\Huge\item\color{blue}\textbf{BITÁCORA DEL DÍA NÚMERO 06 DEL CURSO}}
\author{\Large Mariana Yasmin Martínez García}
\date{\Large 15/01/2019}

\begin{document}
\begin{figure}[t]
	\centering
	\includegraphics[width=0.4\linewidth]{Imagenes/1}
	\caption{Escudo de la Facultad de ciencias}
	\label{fig:1}
\end{figure}
	\maketitle
		
	\newpage
	
	\title{\huge\textbf{Bitácora del día 15 de enero\\}} \\
	El lunes 15 empezamos hablando sobre cómo calcular la raíz cuadrada o aproximar la. Nos ayudamos de un cuadrado de área X (cuyos lados son $\sqrt{X}$) y de un rectángulo de área 1 cuyos lados son h=X y b=1. \\
	Mientras hacíamos esto llegamos a un concepto llamado \textbf{algorítmo} y que el profesor definió como "Es un conjunto de instrucciones para resolver un problema". Llegamos a la conclusión de que para calcular la raíz cuadrada primero h = $\frac{b+h}{2}$ y después b = $\frac{x}{h}$ y que esto se repitiera hasta que |b - h|>e, donde e fuera un error muy pequeñito.\\
	Al llegar a este punto de que una función se repitiera hasta que ocurriera algo fue cuando llegamos a una palabra \textbf{while}, este ciclo nos permite repetir una operación hasta que cierta condición de cumpla.\\ \\
	También se mencionó las palabras \textbf{and} y \textbf{or}, ambas se traducen igual al español, la primera es para poder en listar una serie de condiciones y que todas deben suceder, y la otra te permite que pueda ocurrir una u otra.\\ \\
	Esta clase se usó \textbf{print} que nos permite mostrar el resultado y no solo "almacenarlo"o regresar a él como en el caso de \textbf{return}.\\ \\
	La última parte de la clase nos dedicamos a crear una DIANA (como las del tiro con arco), debíamos colocar la Diana sobre el plano cartesiano y dependiendo de las coordenadas que se le diera debía regresar un valor. \\
	Para este programa usamos \textbf{and}, \textbf{if}, \textbf{else}, \textbf{elif} y \textbf{return}. \\
	El \textbf{elif, if} y \textbf{else} sirven para condicionales. "Si pasa, entonces ocurrirá esto otro". El \textbf{if} sirve para colocar la primer condicional, \textbf{else} nos servirá para condicionar algo si las demás condiciones no ocurrieron y por último \textbf{elif} sirve para cuando vas a colocar más de dos condiciones.\\ \\
	Este programa era un módulo como lo de la clase del viernes por lo que cuando lo quisiéramos ocupar debíamos de importar el módulo con \textbf{import} seguido del nombre del archivo donde está guardado el módulo, posteriormente escribimos el nombre del archivo seguido de un punto y presionamos el tabulador que está a lado de la "Q" en el teclado, esto hacía que se desplegaran las opciones que podíamos ocupar y al seleccionar la que correspondía a lo que estábamos haciendo nos permitía introducir dos valores (uno para X y otro para Y) y al presionar \textbf{enter} aparecía el valor correspondiente a las coordenadas ingresadas. \\ \\
	De tarea se dejó un programa algo parecido, solo que ahora en cierta área del plano cartesiano había un círculo y por lo tanto, ahora había áreas de diferente valor.
	
	
	
	
\end{document}