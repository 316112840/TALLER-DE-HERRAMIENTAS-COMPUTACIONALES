\documentclass[letterpaper, 12pt, twoside]{article}
\usepackage{amsmath}
\usepackage{graphicx}
\usepackage{xcolor}
\usepackage{enumitem}
\usepackage[utf8]{inputenc}
\usepackage{graphicx}

%Aquí inicia la portada del documento
\title{\Huge\item\color{red}\textbf {VITÁCORA DEL DÍA NÚMERO 01 DEL CURSO}}
\author{Mariana Yasmin Martínez García}
\date{13/01/2019}

\begin{document}
	\maketitle
	\begin{figure}
		\centering
		%\includegraphics{/home/mariana/Documentos/TALLER-DE-HERRAMIENTAS-COMPUTACIONALES/Clases/Latex/Imágenes/1}
		\caption{}
		\label{fig:1}
	\end{figure}
	
	
	\newpage
	
	\title{\huge\textbf{Vitácora del día 7 de enero\\}}
	El primer día del curso se nos dió de forma sencilla la forma de evaluación al igual que la introducción de lo que veríamos en el curso.
	Se vió que hay varias distribuciones distintas de \textbf{Linux},estas son:
	\begin{enumerate}
		\item\textbf{Fedora}
		\item\textbf{Ubuntu}

	\end{enumerate} 
	También se nos enseñó algunos comandos, como son: 
	\begin{enumerate}
		\item \textbf{\large pwd }: sirve para visualizar el directorio en el que nos encontramos.
		\item \textbf{\large ls }: este te muestra los archivos que hay en algún directorio.
		\item \textbf{\large ls -a }: este te muestra TODOS los archivos que hay en algún directorio, incluyendo los archivos ocultos.
		\item \textbf{\large ls -l }: te enlista las propiedades y permisos de algún archivo.
		\item \textbf{\large chmod }: con este puedes modificar los permisos de algún archivo.
		\item \textbf{\large mkdir \textit{directorio} }: crea un nuevo directorio.
		\item \textbf{\large cd \textit{directorio}}: se escribe el comando seguido del nombre de un directorio y te dirigirá a este.
		\item \textbf{\large cd .. :} te dirige al directorio donde estabas antes.
		\item \textbf{\large man \textit{comando} }: te muestra el manual del comando que escribiste.	
	\end{enumerate}
    Y al final de la clase se dejó de tarea probar que todos estos comandos sí funcionara.

	
	
	
\end{document}