
\documentclass{article}
\usepackage{amsmath}
\usepackage{amssymb}
\usepackage{graphicx}
\usepackage{enumitem}
\usepackage{xcolor}
\usepackage[utf8]{inputenc}
\graphicspath{{}}	


%Aquí inicia la portada del documento
\title{\Huge\item\color{purple}\textit{VITÁCORA DE LA CLASE NÚMERO 13 DEL CURSO}}
\author{\Large\textit{ Mariana Yasmin Martínez García}}
\date{\Large\textit{ 23/01/2019}}

\begin{document}
\begin{figure}[t]
	\centering
	\includegraphics[width=0.8\linewidth]{Imagenes/1}
	\caption{Escudo de la Facultad de ciencias}
	\label{fig:1}
\end{figure}
	\maketitle
		
	\newpage
	
	\title{\Huge\textbf{\color{purple}\textit{Vitácora: 23 de enero\\}}} \\
	Empezamos esta clase revisando unos problemas del libro \textit{Python fácil}, se habló sobre  dudas de cómo resolverlos y se dejaron de tarea. \\ \\
	Después empezamos a hablar sobre la sucesión de Fibonacci y mientras hacíamos eso, el profesor mencionó "función recursiva" que es aquella que dentro de su bloque se menciona a sí misma y necesita por lo menos un "if", también puede tener varios condicionales y un "else". \\ \\
	Después de realizar un programa que permitiera calcular en enésimo número de la serie de Fibonacci hicimos una función recursiva para sumar los primeros enésimos números positivos naturales. \\ \\
	Cuando terminamos eso, hicimos algo my parecido pero ahora con los objetos de una lista. Cuando hicimos esto el profesor nos mostró que cuando a alguna variable le asignamos el valor 0 o cuando a alguna lista la dejamos vacía, se toma como falso. Sin embargo, cuando la lista tiene objetos o a la variable le asignamos un valor diferente de 0, lo toma como verdadero. Y esto nos sirvió para poder hacer el problema de la lista mediante una función recursiva, esta quedó así:  \\
	\begin{verbatim}
	def printr(L):
		if L: 
		print L[0],
		printr(L[1:])
	\end{verbatim}
	Finalmente revisamos lo que es una variable local y una global. La local es la que está dentro de un bloque, mientras que la global es aquella que puedes usar dentro del bloque donde fue definida. \\
	La forma en la que podemos definir que una variable es global es colocando la palabra reservada \textbf{global} seguido de la variable y de esta manera poder usarla fuera de ese bloque.
\end{document}