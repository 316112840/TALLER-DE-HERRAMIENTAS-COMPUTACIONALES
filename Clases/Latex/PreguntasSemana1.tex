\documentclass[letterpaper, 12pt, twoside]{article}
\usepackage{amsmath}
\usepackage{graphicx}
\usepackage{xcolor}
\usepackage{enumitem}
\usepackage[utf8]{inputenc}
\usepackage{graphicx}

%Aquí inicia la portada del documento
\title{\Huge\item\color{red}\textbf {PREGUNTAS DE LA SEMANA 1 DEL CURSO}}
\author{Mariana Yasmin Martínez García}
\date{13/01/2019}

\begin{document}
	\maketitle
	
	\newpage
	\title{\Large\color{red}\text{DÍA 01}}
	\begin{enumerate}
		\item\text{\Large ¿Qué es Linux?} Linux es un sistema operativo libre, gratuito y estable.
		\item\text{\Large ¿Qúe es una distribución de Linux?} es una versión personalizada del sistema operativo original.
		\item\text{\Large Menciona algunas distribuciones de Linux:} Ubuntu, Fedora y Debian.
		\item\text{\Large ¿Para qué sirve el comando " ls - l "?} te en lista las propiedades y permisos de algún archivo.
		\item\text{\Large ¿Para qué sirve el comando " pwd "?}para saber en qué directorio me encuentro.
		\item\text{\Large ¿Para qué sirve el comando " cd "?} cd seguido de algún directori te permite dirigirte a ese directorio. cd seguido de dos puntos (..) te permite dirigirte al directori en el que te encontrabas anteriormente.
		\item{\Large ¿Con qué comando puedes cambiar los permisos de algún archivo?} con " chmod " seguido del número (dependiendo de los permisos que quieras dar) y posteriormente la ubicación del archivo.
	\end{enumerate}

    \title{\Large\color{red}\text{DÍA 02}} 
    \begin{enumerate}
        \item\text{\Large ¿Qué es Github?} es una plataforma que te permite guardar los archivos en python o algún otro lenguaje de progrmación, documentos en Latex o cualquier otro documento.
        \item\text{\Large ¿Cuál es el manejador de tapetes para Fedora?} dnf
         \item\text{\Large ¿Cuál es el manejador de tapetes para Ubuntu?} apt.
          \item\text{\Large ¿Con qué comandos se instala Git?} \textbf{sudo yum install git} para Fedora y \textbf{sudo apt-get install git} para Ubuntu.
           \item{\Large ¿Con qué comandos se sube los archivos nuevos o las modificaciones a la nube de Git?} \textbf{git add *, git commit, y git push}.
            \item{\Large ¿Con qué comandos se configura Git para poder entrar desde tu cuenta?} con los comandos \textbf{gitconfig --global user.email ""} (entre las comillas se coloca el correo con el que te registraste a Git) y \textbf{gitconfig -- global user.name""} (entre las comillas se coloca el nombre de usuario con el que te registraste).
          
    \end{enumerate}

    \title{\Large\color{red}\text{DÍA 03}} 
    \begin{enumerate}
    	\item{\Large ¿Con qué comando se puede crear on repositorio nuevo desde la terminal?} con el comando \textbf{git init} seguido del nombre del repositorio que se quiere crear.
    	\item{\Large ¿Con qué comando se bajan los archivos nuevos o modificados desde la nube de Git?} con el comando \textbf{git pull}.
    	 \item{\Large ¿Qué comando te permite no tener que abrir el \textbf{vi} para escribir un comentario?} \textbf{git commit -m""} y entre la comillas se coloca el comentario.
    \end{enumerate}

    \title{\Large\color{red}\text{DÍA 04}} 
    \begin{enumerate}
     	\item\text{\Large ¿Qué comandos te permiten descargar el \textbf{idle}?} en Fedora se usa \textbf{dnf install python.tools} y en Ubuntu \textbf{sudo apt idle}
     	\item{\Large ¿Qué comando le ordena al programa mostrar al usuario algo?}\textbf{print}
     	\item{\Large ¿Qué es python?}es un lenguaje de programación.
     	 \item{\Large ¿Qué es una división flotante?} la división flotante únicamente usará la parte entera.
    \end{enumerate}

    \title{\Large\color{red}\text{DÍA 05}} 
    \begin{enumerate}
    	\item\text{\Large ¿Qué es un módulo?} un módulo es una biblioteca, este es un fragmento de algún programa al que puedes llamar después para poder usarlo.
    	\item\text{\Large ¿Qué es una palabra reservada en python?} es una palabra que tiene un significado dentro de python y que por lo tanto no pueden ser usadas como variables.
    	\item\text{\Large ¿Qué comandos se usan para descargar TexStudio?}\textbf{dnf install texstudio} para computadoras con Fedora y \textbf{apt install texstudio} para computadoras con Ubuntu.
    	\item\text{\Large Menciona algunos paquetes en Latex} \textbf{usepackage{xcolor}} sirve para poder cambiar de color cierta parte del texto, \textbf{usepackage{amsmath}} sirve para poder incluir símbolos matemáticos, \textbf{usepackage[utf8]{inputenc}} sirve para poder usar acentos y diéresis sin que te lo marque como error y \textbf{usepackage{graphicx}} sirve para poder incluir imágenes dentro del documento.
    \end{enumerate}

\end{document}