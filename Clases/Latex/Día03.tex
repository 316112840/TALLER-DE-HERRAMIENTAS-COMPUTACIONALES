\documentclass[letterpaper, 12pt, twoside]{article}
\usepackage{amsmath}
\usepackage{graphicx}
\usepackage{xcolor}
\usepackage{enumitem}
\usepackage[utf8]{inputenc}
\usepackage{graphicx}

%Aquí inicia la portada del documento
\title{\Huge\item\color{red}\textbf {VITÁCORA DEL DÍA NÚMERO 03 DEL CURSO}}
\author{Mariana Yasmin Martínez García}
\date{13/01/2019}

\begin{document}
	\maketitle
	
	\newpage
	
	\title{\huge\textbf{Vitácora del día 9 de enero\\}}
	Este día se siguió viendo algunos comandos nuevos en \textbf{git}. Por lo que para mí hizo que me quedaran mucho más claro sus funciones:
	\begin{enumerate}
		\item\textbf{\large git pull}: si hiciste algún cambio en otro dispositivo, este comando te permitirá actualizar y bajar los nuevos cambios hechos.
		\item\textbf{\large git status}: te permitirá ver los cambios que has hecho y que aún no son subidos a la nube.
		\item\textbf{\large git commit -m""}: este comando sirve para evitar abrir el \textbf{vi}, únicamente deberás escribir el comentario dentro de las comillas.
		\item\textbf{\large git init}: en lugar de entrar a la página desde el navegador y crear el repositorio ahí, podemos usar este comando que nos permitirá crear un repositorio vacío.
	\end{enumerate}
    Aparte de estos nuevos comandos, repasamos algunos como \textbf{git push, git add *,git commit}, y en mi caso (porque cambié de computadora), \textbf{git clone,gitconfig --global user.email ""}y \textbf{gitconfig --global user.name ""}. \newline
	Este día también se nos enseñó que el símbolo "mayor que" nos permite redireccionar la salida de algún comando. Nosotros colocamos \textbf{history > Clases/Latex/Comandos03.txt} esto nos permitió guardar el historial de comandos usados en un archivo que creó llamado \textbf{Comandos03}. \\
	Y por último el profesor definió la forma de organización de nuestro repositorio. Así que dentro de él hicimos una carpeta llamada \textbf{Clases} y entro de esta colocamos dos carpetas, una llamada \textbf{Látex} y la otra \textbf{Programas}.
	
\end{document}