\documentclass{article}
\usepackage{amsmath}
\usepackage{amssymb}
\usepackage{graphicx}
\usepackage{enumitem}
\usepackage{xcolor}
\usepackage[utf8]{inputenc}
\graphicspath{{}}	


%Aquí inicia la portada del documento
\title{\Huge\item\color{purple}\textit{VITÁCORA DE LA CLASE NÚMERO 14 DEL CURSO}}
\author{\Large\textit{ Mariana Yasmin Martínez García}}
\date{\Large\textit{ 24/01/2019}}

\begin{document}
\begin{figure}[t]
	\centering
	\includegraphics[width=0.8\linewidth]{Imagenes/1}
	\caption{Escudo de la Facultad de ciencias}
	\label{fig:1}
\end{figure}
	\maketitle
		
	\newpage
	
	\title{\Huge\textbf{\color{purple}\textit{Vitácora: 24 de enero\\}}} \\
	Este día revisamos dudad que teníamos de los problemas que se nos dejó la clase anterior, en específico no centramos en el problema 5 que es sobre encontrar la salida de un laberinto dándonos las coordenadas de la entrada. \\
	Revisamos paso por paso lo que debíamos hacer. Empezamos haciendo un laberinto trivial en el que la salida estuviera unos cuantos pasos frente a la entrada, donde no debíamos subir, bajar o regresar, solo debíamos seguir avanzando. La forma para saber si ya estábamos en la salida es que no hubiera ninguna obstrucción y que estuviéramos en el límite del laberinto, es decir, que si el laberinto era de m*n la primer coordenada del la posición donde estuviéramos fuera igual a m. \\
	Después lo hicimos un poco más difícil colocado la salida en la parte de abajo. Y así quedó finalmente el código:  \\
	\begin{verbatim}
	def resolver(L, e):
		n = len(L[0]) 
		m = len(L)
		x = e[0]
		y = e[1]
		if y == n-1 or x==m-1:
			return e[0]+1, e[1]+1 
		else:
			if L[x][y+1] == False:
				e = [x, y+1]
				return resolver(L, e) 
			elif L[x + 1][y] == False:
				e = [x+1, y]
				return resolver(L, e) 
			else:
				print "Ya no se puede seguir avanzando"
	\end{verbatim}
	Y terminar el resto se dejó de tarea. \\ \\
	Cuando terminamos de hacer eso nos enseñaron que al usar \textbf{list(' ')} y colocando entre las comillas alguna palabra o algún número, este se regresará en una lista dividiendo a cada uno de los caractéres por comillas, es decir:
	\begin{verbatim}
	>>> list('Mi nombre es Mariana')
	['M', 'i', ' ', 'n', 'o', 'm', 'b', 'r', 'e', ' ', 'e', 's', ' ',
	 'M', 'a', 'r', 'i', 'a', 'n', 'a']
	\end{verbatim}
	  

\end{document}