\documentclass[letterpaper, 12pt, twoside]{article}
\usepackage{amsmath}
\usepackage{graphicx}
\usepackage{xcolor}
\usepackage{enumitem}
\usepackage[utf8]{inputenc}
\usepackage{graphicx}

%Aquí inicia la portada del documento
\title{\Huge\item\color{red}\textbf {VITÁCORA DEL DÍA NÚMERO 04 DEL CURSO}}
\author{Mariana Yasmin Martínez García}
\date{13/01/2019}

\begin{document}
	\maketitle
	
	\newpage
	\title{\huge\textbf{Vitácora del día 10 de enero\\}} \\
	El día 10 de enero del 2019 empezamos a usar \textbf{python}. \\
	Pero antes de esto, el profesor escogió un problema y la primer parte de la clase la utilizamos en entender el problema, pensar qué nos pedía y cómo poder resolverlo. Para que de esta manera, al ya saber qué estábamos haciendo, pudiéramos empezar a pasar el problema a la computadora. \\ \\
	Y para esto usamos un entorno de programación (IDE) llamado \textbf{idle}. Para poder descargarlo se debe usar \textbf{dnf install python.tools} para equipos con la distribución Fedora y \textbf{sudo apt idle} para equipos con Ubuntu. \\ \\
	Al colocar \textbf{idle} sobre la terminal se abrirá un \textbf{SHELL}, que es un interprete de comandos especializado. Aquí empezamos a hacer algunas pruebas sobre operaciones básicas. Dando le órdenes a \textbf{python} como \textbf{"print  34*3 - 1/2 *9.81*3**2"}, al colocar esto nos dimos cuenta de que el resultado era erróneo y esto se debió al hacer la división \textbf{1/2} solo tomaba la parte entera y para que también tomara los decimales debíamos colocar \textbf{1.0/2}, o bien, \textbf{1/2.0}. De esta forma, al colocar \textbf{"print  34*3 - 1.0/2 *9.81*3**2"} o \textbf{"print  34*3 - 1/2.0 *9.81*3**2"} el resultado ya era correcto.\\
	Al terminar de hacer estas pruebas, sobre el \textbf{SHELL} dimos click sobre \textbf{New file} que se encontraba en la parte superior izquierda,con esto se abrió un nueva pestaña donde empezamos a colocar los valores del problema que el profesor escogió. Colocamos la fórmula que resolvía el problema (en este caso era un problema sobre tiro parabólico), ¡Y listo! Ya habíamos resuelto el problema. \\ \\
	En el resto de la clase cada quien escogió un problema y lo resolvió de la misma manera. \\
	Yo escogí un problema sobre tiro vertical y como este me pedía usar raíz cuadrada necesité importar una \textbf{biblioteca} con el comando: \textbf{import math } y posteriormente ya podía colocar la operación \textbf{math.squirt ()},dentro de los paréntesis se coloca a lo que se le quiera calcular la raíz cuadrada.
	

\end{document}