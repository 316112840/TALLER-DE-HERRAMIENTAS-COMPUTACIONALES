\documentclass{article}
\usepackage{amsmath}
\usepackage{amssymb}
\usepackage{graphicx}
\usepackage{enumitem}
%\usepackage{xcolor}
\usepackage[dvipsnames]{xcolor}
\usepackage[utf8]{inputenc}
\graphicspath{{}}	


%Aquí inicia la portada del documento
\title{\Huge\item\color{magenta}\textit{PREGUNTAS DE LA CLASE 14 DEL CURSO}}
\author{\Large\textit{Mariana Yasmin Martínez García}}
\date{\Large\textit{24/01/2019}}


\begin{document}
\begin{figure}[t]
	\centering
	\includegraphics[width=0.7\linewidth]{Imagenes/1}
	\caption{Escudo de la Facultad de ciencias}
	\label{figura:1}
\end{figure}

	\maketitle
		
	\newpage
	
	\title{\huge\color{magenta}\textbf{\textit{Cuestionario de la clase 14}}} \\
	\begin{enumerate}
		 
		 \item{\Large\color{purple} ¿Qué es una matriz en Python? } Es una lista que contiene listas del mismo tamaño.
		 \item{\Large\color{purple} ¿Cómo se usaron las matrices para definir el laberinto? } El laberinto es una matriz, y cada una de las listas que contiene tienen \textbf{true} o \textbf{false}, donde hay un false significa que se puede avanzar y donde hay un true es porque hay una obstrucción. 
		 \item{\Large\color{purple} ¿Qué hace \textbf{list(' ') ? }} lo que esté dentro de las comillas se colocará en una lista pero separará cada uno de los caractéres de los que haya escrito por medio de comillas.
		  \item{\Large\color{purple} ¿Qué es una tupla? } es una lista que no puede ser modificada.
		   \item{\Large\color{purple} En el problema del laberinto, ¿De qué forma se podía regresar el resultado en forma de tupla y cómo lograr que fuera en forma de lista?} dentro del bloque if se colocó \textbf{return e[0]+1, e[1]+1} de esta manera el resultado se mostraba en forma de tupla, si hubiéramos escrito \textbf{return [e[0]+1, e[1]+1]} el resultado estaría en forma de lista.
	\end{enumerate}
	
\end{document}