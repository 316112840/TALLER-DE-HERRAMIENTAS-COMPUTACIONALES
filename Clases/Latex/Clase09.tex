\documentclass{article}
\usepackage{amsmath}
\usepackage{amssymb}
\usepackage{graphicx}
\usepackage{enumitem}
\usepackage{xcolor}
\usepackage[utf8]{inputenc}
\graphicspath{{}}	


%Aquí inicia la portada del documento
\title{\Huge\item\color{blue}\textbf{VITÁCORA DEL DÍA NÚMERO 09 DEL CURSO}}
\author{\Large Mariana Yasmin Martínez García}
\date{\Large 17/01/2019}

\begin{document}
\begin{figure}[t]
	\centering
	\includegraphics[width=0.7\linewidth]{Imagenes/1}
	\caption{}
	\label{1}
\end{figure}

	\maketitle
		
	\newpage
	
	\title{\huge\color{blue}\textbf{Vitácora del día 17 de enero\\}} \\
	En esta clase empezamos con LaTex.\\
	Lo primero que hicímos fue terminar de hacer el documento de la clase pasada, aprendimos a hacer tablas y a colocar secciones del texto alineado en el centro.\\ \\
	Todas las clases pasadas habíamos trabajado con \textbf{documentclass{article}} pero en esta clase, después de terminar con el Ejercicio 2 (que empezamos la clase pasada), creamos otro documento con \textbf{documentclass{book}}. Este nos permite dar formato de libro al documento.\\
	Dentro de este nuevo documento usamos \textbf{usepackage{hyperref}} que nos permite enlazar cosas, como enlazar el contenido a la sección o capítulo que se quiera.\\
	Con \textbf{tableofcontents} se hace una tabla de contenidos del texto, es decir, coloca cada una de las secciones o capitulos que hay en todo el documento. \\
	Con \textbf{chapter{ }]} se hace el capítulo, escribiendo dentro de las llaves el nombre de como queramos que se llame.\\
	Con \textbf{begin{verbatim}} y cerrando con \textbf{end{verbatim}} y colocando entre ambos algún texto, este texto se colocará tal y cómo está, lo que nos permitiría introducir códigos al documento.\\
	Con \textbf{input} seguido de la ubicación de algún archivo, nos permite introducir el contenido de este archivo al documento. Sin embargo, en la clase intentamos introducir un archivo de Pyhton para poder incluir el código de ese archivo y cuando queríamos compilar el documento nos marcaba error, y con otro documento que creamos al que llamamos "Prueba" sí nos dejó. \\ 
	Lo último que hicimos en Látex fue hacer una bibliografía con \textbf{beginthebibliography} y \textbf{endethebibliography}. \\ \\
	Terminando de ver lo correspondiente a LaTex, empezamos a revisar uno de los programas que tuvimos de tarea, el que servía para calcular el máximo común divisor. \\
	Lo resolvimos con el método del algoritmo de Euclides.\\
	Mientras revisábamos esto, aprendimos que \textbf{!=} sirve para desigualdades, si escribo \textbf{a != b} quiere decir "a no es igual a b".\\
	El \textbf{\%} sirve para módulos, si escribo \textbf{a\% b} quiere decir "el residuo de dividir a sobre b". \\  \'
	También usamos \textbf{from     import     } después del from colocamos el nombre del módulo y después del import colocamos la función que queramos importar o usar del módulo. \\
	El profesor nos explicó que era mejor importar únicamente lo que fuéramos a usar ara evitar que el programa usara tanta memoria. \\
	Y finalmente, esto lo usamos también al final de la clase con: \textbf{from os import getcwd as pwd, listdir as ls, chdir as cd} y esto lo hicimos con la intención de cambiar los comandos del shell a los que usamos en el Bash, para de esta manera poder usar comandos que ya habíamos usado.\\

\end{document}