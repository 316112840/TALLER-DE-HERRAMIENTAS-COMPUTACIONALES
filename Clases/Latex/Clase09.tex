\documentclass{article}
\usepackage{amsmath}
\usepackage{amssymb}
\usepackage{graphicx}
\usepackage{enumitem}
\usepackage{xcolor}
\usepackage[utf8]{inputenc}
\graphicspath{{}}	


%Aquí inicia la portada del documento
\title{\Huge\item\color{blue}\textbf{VITÁCORA DEL DÍA NÚMERO 09 DEL CURSO}}
\author{\Large Mariana Yasmin Martínez García}
\date{\Large 17/01/2019}

\begin{document}
\begin{figure}[t]
	\centering
	\includegraphics[width=0.7\linewidth]{Imagenes/1}
	\caption{}
	\label{fig:1}
\end{figure}

	\maketitle
		
	\newpage
	
	\title{\huge\textbf{Vitácora del día 17 de enero\\}} \\
	En esta clase empezamos con LaTex.\\
	Lo primero que hicímos fue terminar de hacer el documento de la clase pasada, aprendimos a hacer tablas y a colocar secciones del texto alineado en el centro.\\ \\
	Todas las clases pasadas habíamos trabajado con \textbf{documentclass{article}} pero en esta clase, después de terminar con el Ejercicio 2 (que empezamos la clase pasada), creamos otro documentos con \textbf{documentclass{book}}. Este nos permite dar formato de libro al documento.\\
	Dentro de este nuevo documento usamos \textbf{usepackage{hyperref}} que nos permite enlazar cosas.\\
	Con \textbf{tableofcontents} se hace un contenido del texto. \\
	Con \textbf{chapter{ }]} se hace el caítulo, escribiendo dentro de las llaves el nombre de como queramos que se llame.\\
	Con \textbf{begin{verbatim}} y cerrando con \textbf{end{verbatim}} y colocando entre ambos algún texto, este texto se colocará



\end{document}