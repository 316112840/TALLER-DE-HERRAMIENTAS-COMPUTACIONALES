\documentclass{article}
\usepackage{amsmath}
\usepackage{amssymb}
\usepackage{graphicx}
\usepackage{enumitem}
%\usepackage{xcolor}
\usepackage[dvipsnames]{xcolor}
\usepackage[utf8]{inputenc}
\graphicspath{{}}	


%Aquí inicia la portada del documento
\title{\Huge\item\color{magenta}\textit{PREGUNTAS DE LA CLASE 12 DEL CURSO}}
\author{\Large\textit{Mariana Yasmin Martínez García}}
\date{\Large\textit{21/01/2019}}


\begin{document}
\begin{figure}[t]
	\centering
	\includegraphics[width=0.7\linewidth]{Imagenes/1}
	\caption{Escudo de la Facultad de ciencias}
	\label{figura:1}
\end{figure}

	\maketitle
		
	\newpage
	
	\title{\huge\color{magenta}\textbf{\textit{Cuestionario de la clase 12}}} \\
	\begin{enumerate}
		 \item{\Large\color{red} ¿Para qué sirve "pprint"?} Sirve para imprimir de una manera más organizada y bonita.
		 \item{\Large\color{red} ¿Cuál es la diferencia entre hacer una copia de una lista y hacer una referencia de ella?} Cuando se hace una copia y se modifica la lista original o la copia, la otra no se ve afectada, únicamente modificas esa. Sin embargo, cuando haces ua referencia  modificas cualquiera de la dos, ambas se modifican.
		 \item{\Large\color{red} ¿Menciona una forma de hacer una copia de una lista?} La forma que vimos en clase es creando una lista nueva en la que se incluyeran todos los elementos de la otra lista.
		 \item{\Large\color{red} ¿Qué se usa para recorrer una lista por índices?} se usa \textbf{for i in range( )} donde i puede ser sustituida por cualquier otra variable
		 \item{\Large\color{red} ¿Para qué sirve "documentclass{beamer}"?} Sirve para crear presentaciones tipo PowerPoint.
	\end{enumerate}
	
\end{document}